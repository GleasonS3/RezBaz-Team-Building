\section{Kaity's Section}


When we look to the individuals of the same variety or sub-variety of our older cultivated plants and animals, one of the first points which strikes us, is, that \textbf{they generally differ much more from each other}, than do the individuals of any one species or variety in a state of nature.\cite{Fisher_2012}

When we reflect on the vast diversity of the plants and animals which have been cultivated, and which have varied during all ages under the most different climates and treatment, I think we are driven to conclude that \textit{this greater variability is simply due to our domestic productions having been raised under conditions of life not so uniform as, and somewhat different from, those to which the parent-species have been exposed under nature.}

There is, also, I think, some probability in the view propounded by Thomas Andrew Knight (\href{http://en.wikipedia.org/wiki/Main_Page}{Wikipedia}), that this variability may be partly connected with excess of food. It seems pretty clear that

\begin{enumerate}
\item organic beings must be exposed during several generations to the new conditions of life to cause any appreciable amount of variation;
\item that when the organisation has once begun to vary, it generally continues to vary for many generations.
\end{enumerate}

No case is on record of a variable being ceasing to be variable under cultivation. Our oldest cultivated plants, such as wheat, still often yield new varieties: our oldest domesticated animals are still capable of rapid improvement or modification.